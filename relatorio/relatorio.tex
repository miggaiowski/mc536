\documentclass[12pt,a4paper]{article}
 % document definitions
\input{settings}
\begin{document}
% include cover

\begin{titlepage}

% command aliases
\newcommand{\HRule}{\rule{\linewidth}{0.5mm}} 
 
\begin{center}
 
% Upper part of the page
%\includegraphics[width=0.15\textwidth]{./logo}\\[1cm]

%\HRule \\
\textsc{\LARGE Universidade Estadual de Campinas}\\[0.5cm]
\textsc{\Large Banco de Dados - MC536}
%\HRule \\
{\ }\\[4.5cm]
 
% Title
%{\ }\\[2.0cm]
{ \huge \bfseries Projeto Lógico do Banco de Dados}\\[3.0cm]
 
% Student and advisor
\begin{flushright}
Miguel Francisco Alves de Mattos Gaiowski \\
\emph{RA 076116} \\
Luiz Claudio Carvalho\\
\emph{RA 800578} \\[2.0cm]
\end{flushright}


\begin{flushright}
Prof. Ricardo Torres \\
\end{flushright}
 
\vfill
 
% Bottom of the page
{\large Campinas, \today}
 
\end{center}
 
\end{titlepage}

\tableofcontents
\listoffigures

%%%%%%%%%%%%%%%%%%%%%%%%%%%%
\section{Introdução}
%%%%%%%%%%%%%%%%%%%%%%%%%%%%

Este relatório descreve a etapa final do trabalho prático de MC536: a implementação 

%%%%%%%%%%%%%%%%%%%%%%%%%%%%
\section{Implementação}
%%%%%%%%%%%%%%%%%%%%%%%%%%%%

Utilizamos o framework Django \cite{django}, em Python, para fazer toda a interface do usuário
com o banco de dados. Trata-se um sistema que facilita a implementação de aplicativos web que
interagem com bancos da dados.

Para utilizar a ferramenta, precisamos descrever os esquemas de relação para que fosse gerado
o código SQL e executado no SGBD escolhido.

Com a modelagem dos esquemas de relação da fase 2, ficou mais fácil modelar as classes que o Django
precisa. Escolhemos utilizar o SGDB SQLite3, que mostrou-se mais prático já que é acompanha a instalação
do Python e todo o banco fica em arquivo, facilitando o desenvolvimento de um projeto pequeno como este.

Abaixo segue o código SQL de criação das tabelas.

\begin{verbatim}
BEGIN;
CREATE TABLE "linguas_programa" (
    "id" integer NOT NULL PRIMARY KEY,
    "nome" varchar(200) NOT NULL,
    "organizacao_id" varchar(30) NOT NULL
)
;
CREATE TABLE "linguas_autor" (
    "id" integer NOT NULL PRIMARY KEY,
    "nome" varchar(200) NOT NULL,
    "nacionalidade" varchar(50) NOT NULL,
    "data_nasc" datetime NOT NULL
)
;
CREATE TABLE "linguas_redetrabalho" (
    "nome" varchar(200) NOT NULL PRIMARY KEY
)
;
CREATE TABLE "linguas_programa_do_autor" (
    "id" integer NOT NULL PRIMARY KEY,
    "autor_id" integer NOT NULL REFERENCES "linguas_autor" ("id"),
    "programa_id" integer NOT NULL REFERENCES "linguas_programa" ("id")
)
;
CREATE TABLE "linguas_idioma" (
    "id" integer NOT NULL PRIMARY KEY,
    "nome" varchar(200) NOT NULL,
    "pag_wiki" varchar(200) NOT NULL
)
;
CREATE TABLE "linguas_local_do_idioma" (
    "id" integer NOT NULL PRIMARY KEY,
    "idioma_id" integer NOT NULL REFERENCES "linguas_idioma" ("id"),
    "pais" varchar(200) NOT NULL,
    "regiao" varchar(200) NOT NULL
)
;
CREATE TABLE "linguas_keyword" (
    "texto" varchar(50) NOT NULL PRIMARY KEY,
    "idioma_id" integer NOT NULL REFERENCES "linguas_idioma" ("id")
)
;
CREATE TABLE "linguas_natureza" (
    "tipo" varchar(200) NOT NULL PRIMARY KEY
)
;
CREATE TABLE "linguas_usuario" (
    "login" varchar(30) NOT NULL PRIMARY KEY,
    "nome" varchar(200) NOT NULL,
    "email" varchar(75) NOT NULL,
    "senha" varchar(20) NOT NULL,
    "tipo" varchar(30) NOT NULL,
    "pais_sede" varchar(200) NOT NULL
)
;
CREATE TABLE "linguas_redeusuario" (
    "id" integer NOT NULL PRIMARY KEY,
    "login_id" varchar(30) NOT NULL REFERENCES "linguas_usuario" ("login"),
    "nome_id" varchar(200) NOT NULL REFERENCES "linguas_redetrabalho" ("nome")
)
;
CREATE TABLE "linguas_documento" (
    "id" integer NOT NULL PRIMARY KEY,
    "arquivo" varchar(100) NOT NULL,
    "formato" varchar(5) NOT NULL,
    "titulo" varchar(200) NOT NULL,
    "data_criacao" datetime NOT NULL,
    "data_submissao" datetime NOT NULL,
    "num_vizualizacoes" integer NOT NULL,
    "autor_id" integer NOT NULL REFERENCES "linguas_autor" ("id"),
    "idioma_id" integer NOT NULL REFERENCES "linguas_idioma" ("id"),
    "submetido_por_id" varchar(30) NOT NULL REFERENCES "linguas_usuario" ("login"),
    "maquina_submissao" char(15) NOT NULL,
    "programa_id" integer NOT NULL REFERENCES "linguas_programa" ("id"),
    "natureza_id" varchar(200) NOT NULL REFERENCES "linguas_natureza" ("tipo")
)
;
CREATE TABLE "linguas_descricao" (
    "id" integer NOT NULL PRIMARY KEY,
    "idioma_id" integer NOT NULL REFERENCES "linguas_idioma" ("id"),
    "texto" varchar(1000) NOT NULL
)
;
CREATE TABLE "linguas_documentokeyword" (
    "id" integer NOT NULL PRIMARY KEY,
    "documento_id" integer NOT NULL REFERENCES "linguas_documento" ("id"),
    "texto_id" varchar(50) NOT NULL REFERENCES "linguas_keyword" ("texto")
)
;
CREATE TABLE "linguas_favorito" (
    "id" integer NOT NULL PRIMARY KEY,
    "documento_id" integer NOT NULL REFERENCES "linguas_documento" ("id"),
    "login_id" varchar(30) NOT NULL REFERENCES "linguas_usuario" ("login")
)
;
CREATE TABLE "linguas_responde" (
    "id" integer NOT NULL PRIMARY KEY,
    "sobre_id" integer NOT NULL REFERENCES "linguas_documento" ("id"),
    "resposta_id" integer NOT NULL REFERENCES "linguas_documento" ("id")
)
;
CREATE TABLE "linguas_liga" (
    "id" integer NOT NULL PRIMARY KEY,
    "login_id" varchar(30) NOT NULL REFERENCES "linguas_usuario" ("login"),
    "did_a_id" integer NOT NULL REFERENCES "linguas_documento" ("id"),
    "did_liga_id" integer NOT NULL REFERENCES "linguas_documento" ("id")
)
;
CREATE TABLE "linguas_debate" (
    "id" integer NOT NULL PRIMARY KEY,
    "comentario_1_id" integer NOT NULL,
    "comentario_2_id" integer NOT NULL
)
;
CREATE TABLE "linguas_comentario" (
    "id" integer NOT NULL PRIMARY KEY,
    "usuario_id" varchar(30) NOT NULL REFERENCES "linguas_usuario" ("login"),
    "documento_id" integer NOT NULL REFERENCES "linguas_documento" ("id"),
    "idioma_id" integer NOT NULL REFERENCES "linguas_idioma" ("id"),
    "data" datetime NOT NULL,
    "texto" varchar(1000) NOT NULL
)
;
COMMIT;
\end{verbatim}

Este código foi gerado pelo framework através da descrição em Python dada.

Obviamente que citamos aqui apenas a parte referente ao banco de dados, já que este é o intuito da disciplina. 
Deixamos de mostrar aqui (mas está sendo entregue) o código de toda a interface e das consultas.

%%%%%%%%%%%%%%%%%%%%%%%%%%%%
\section{Casos de Uso}
%%%%%%%%%%%%%%%%%%%%%%%%%%%%

Dado o curto período de tempo que tivemos para fazer a implementação, não foi possível ter todas as funcionalidades
desejadas no sistema. Sendo assim, resolvemos implementar algumas que julgamos fundamentais para demonstrar os conhecimentos
da disciplina.

Temos uma interface de administração que permite inserir e editar ``objetos'' no banco de dados. Por exemplo, pode-se
criar novos usuários, inserir novos documentos, remover comentários inadequados, entre várias outras coisas.

A página principal já mostra uma lista com os cinco últimos documentos adicionados no banco, como pode ser visto na figura \ref{fig:home}.

\begin{figure}[h]
  \centering
    \includegraphics[width=1.0\textwidth]{home.png}
    \caption{Página Inicial}
    \label{fig:home}
\end{figure}

Na interface de usuário, fizemos questão de mostrar os metadados de um documento. Pode-se listar todos documentos
cadastrados no banco \ref{fig:listadocs}, e clicando sobre um deles ver seus detalhes \ref{fig:detalhedocs}.

\begin{figure}[h]
  \centering
    \includegraphics[width=1.0\textwidth]{listadocs.png}
    \caption{Lista com todos documentos cadastrados no banco}
    \label{fig:listadocs}
\end{figure}

\begin{figure}[h]
  \centering
    \includegraphics[width=1.0\textwidth]{detalhedocs.png}
    \caption{Detalhes de um documento}
    \label{fig:detalhedocs}
\end{figure}

Claro que um usuário não quer somente ter uma lista com os documentos cadastrados, ele quer na verdade fazer 
consultas e ter resultados mais específicos. Para tanto implementamos um sistema de buscas. O usuário tem um
campo onde digita uma palavra, ou mais de uma ou até mesmo só parte dela. O sistema busca o termo nos títulos
de documentos, nos nomes de usuários cadastrados e nos nomes de programas patrocinadores. Os resultados são 
categorizados e exibidos para o usuário. Clicando sobre um dos resultados ele é encaminhado para a página de 
detalhes. Pode-se ver um exemplo de uso na figura \ref{fig:busca}.

\begin{figure}[h]
  \centering
    \includegraphics[width=1.0\textwidth]{busca.png}
    \caption{O resultado de uma busca}
    \label{fig:busca}
\end{figure}

Na página de detalhes de um documento o usuário pode ver os comentários já feitos sobre aquele documento
e deixar o seu comentário também. Ver figura \ref{fig:comenta}.

\begin{figure}[h]
  \centering
    \includegraphics[width=1.0\textwidth]{comenta.png}
    \caption{Usuário deixando um comentário}
    \label{fig:comenta}
\end{figure}

Documentos também podem ser procurados por Idioma, por Programa patrocinador ou por Keyword.

%%%%%%%%%%%%%%%%%%%%%%%%%%%%
\section{Bibliografia}
%%%%%%%%%%%%%%%%%%%%%%%%%%%%

\nocite{elmasri}
\nocite{indiano}

\bibliographystyle{plain}
\bibliography{bibliography}

\end{document}


